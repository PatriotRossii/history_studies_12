\documentclass{article}
\usepackage[utf8]{inputenc}

\usepackage[T2A]{fontenc}
\usepackage[utf8]{inputenc}
\usepackage[russian]{babel}

\usepackage{multienum}
\usepackage{geometry}
\usepackage{hyperref}

\geometry{
    left=1cm,right=1cm,
    top=2cm,bottom=2cm
}

\usepackage{graphicx}
\graphicspath{ {./images/} }

\title{История}
\author{Лисид Лаконский}
\date{May 2023}

\newtheorem{definition}{Определение}

\begin{document}
\raggedright

\maketitle
\tableofcontents
\pagebreak

\section{Практическое занятие по истории №13, «СССР в канун Второй мировой войны»}

\subsection{Нарастание напряженности в Европе после прихода в 1933 г. в Германии к власти НСДАП}

30 января 1933 года в Германии к власти приходит А. Гитлер, лидер НСДАП. Нацисты, взяв за основу труд Гитлера «Майн кампф» (1925) с его идеологией завоевания для арийской расы «жизненного пространства», \textbf{принялись активно готовить базу для реализации конкретных текущих и среднесрочных внешнеполитических задач}.

\hfill

Для нацистов было принципиально важно постепенно освободиться от сдерживавших Германию положений Версальской системы и тем самым проложить дорогу для ничем не ограниченной подготовки к войне. Для этого \textbf{германское руководство успешно противодействовало созданию в Европе системы коллективной безопасности}. Германская пропаганда утверждала, что поскольку Германия является «щитом против большевистской опасности и защитницей западной цивилизации», то ей требуются адекватные вооружённые силы. «Равноправный» уровень вооружений являлся на деле лишь прикрытием для начального этапа создания мощной боевой армии.

\hfill

Правительства западных стран отказались осудить односторонние действия Гитлера и в ряде заявлений выразили свою готовность вести переговоры с Германией вне рамок конференции по разоружению и Лиги наций. Тем самым идее создания системы коллективной безопасности был нанесён непоправимый удар.

\hfill

Уже \textbf{14 октября 1933 года Германия объявила о выходе из Лиги наций}. В январе 1935 года в результате плебисцита Германии был возвращён \textbf{Саар}, который до этого находился под протекторатом Лиги наций, а в марте Гитлер заявил о \textbf{разрыве Версальского договора и о восстановлении всеобщей воинской повинности}, то есть о создании регулярной армии рейха — вермахта, включая люфтваффе. 18 июня 1935 года было заключено \textbf{англо-германское морское соглашение}, что явилось двусторонним нарушением Версальского мирного договора. Таким образом Германия получала право довести общий тоннаж своего флота до 420 595 длинных тонн.

\hfill

В 1936 году \textbf{германская армия вступила в демилитаризованную Рейнскую область}. Ремилитаризация Рейнской области позволила занять удобный плацдарм для последующего военного удара по западным соседям Германии — Франции, Бельгии, Голландии и Люксембургу.

На западных границах Третьего рейха возводился Западный вал («линия Зигфрида»), \textbf{сооружались военные укрепления}, мосты и автострады, ведущие к границам соседних стран. В том же году в связи с гражданской войной в Испании была создана \textbf{ось «Берлин — Рим»} и заключён \textbf{Антикоминтерновский пакт с Японией}.

\textbf{24 августа 1936 г.} был опубликован закон о продлении срока службы в германской армии с одного до двух лет. К концу 1936 года в Германии насчитывалось 14 армейских корпусов и одна кавалерийская бригада. Регулярная армия достигла численности 700—800 тыс. человек. В 1936 году Германия имела уже не менее 1500 танков. Её промышленность выпускала более 100 танков в месяц. Громадные средства затрачивались и на создание авиации. В 1936 году германский военно-воздушный флот насчитывал 4500 самолётов. По всей Германии была развёрнута широкая сеть аэродромов, число которых превышало 400. В 1939 году сухопутные войска Третьего рейха насчитывали 2,6 млн человек, ВВС — 400 тыс., ВМФ — 50 тыс. человек.

\pagebreak
\subsection{Проба сил перед столкновением. Гражданская война в Испании и великие державы}

Общая информация по ссылке: \url{https://ria.ru/20160718/1466897981.html}

\hfill

\textbf{Гражданская война в Испании} (1936-1939) — вооруженный конфликт на основе социально-политических противоречий между лево-социалистическим (республиканским) правительством страны, поддерживаемым коммунистами, и поднявшими вооруженный мятеж право-монархическими силами, на сторону которых встала большая часть испанской армии во главе с генералиссимусом Франсиско Франко.

\hfill

Последних поддержали фашистская Италия и нацистская Германия, на стороне республиканцев выступил СССР и добровольцы-антифашисты из многих стран мира. Война закончилась установлением военной диктатуры Франко.

\hfill

Политика советского руководства во главе со Сталиным в отношении Испании менялась по мере того, как изменялась обстановка. До середины сентября 1936 года \textbf{Сталин не планировал какого-либо вмешательства во внутренние дела Испании}. Напротив, по его указанию советские дипломаты, прежде всего во Франции, получили жёсткое указание отклонять все просьбы представителей Испанской республики о советской военной помощи. \textbf{СССР присоединился к соглашению о «невмешательстве в испанские дела»}, предложенному Великобританией и Францией.

\hfill

По мере того, как Великобритания и Франция искали компромисса с гитлеровской Германией и фашистской Италией, \textbf{Сталин стал рассматривать Испанию как потенциального союзника и решил оказать ей помощь}. При этом советским специалистам, сопровождавшим проданное вооружение, давалась инструкция строжайшим образом не вмешиваться во внутренние дела республики.

\hfill

Однако с мая 1937 года, после барселонских событий, \textbf{Сталин стал активно вмешиваться в политическую борьбу в Испании}. Главным объектом преследований советских спецслужб и пропагандистского аппарата стала троцкистская \textbf{Рабочая партия марксистского объединения}, занимавшая антисталинские позиции.

\hfill

До 1938 года \textbf{советская помощь республиканцам уравновешивала материально-техническое вмешательство Германии и Италии}, но в конце 1937 года \textbf{советская помощь стала уменьшаться, в то время как вмешательство Германии и Италии возрастало}. Ослабление советской помощи было связано как с разочарованием советского руководства в новом правительстве Хуана Негрина, которое не могло добиться обещанного перелома в войне, так и с тем, что \textbf{испанская проблема становилась менее важной по сравнению с кризисами в Китае и в Чехословакии}.

\pagebreak
\subsection{Сближение СССР и Германии. Пакт Риббентропа — Молотова и раздел Европы}

\textbf{Германо-советский пакт} представлял собой договор, подписанный нацистской Германией и Советским Союзом \textbf{23 августа 1939 года}. Он был согласован немецким и советским министрами иностранных дел — Иоахимом фон Риббентропом и Вячеславом Молотовым. Германо-советский пакт, или пакт Молотова-Риббентропа, также известен как «нацистско-советский пакт» или «пакт Гитлера-Сталина».

\hfill

Пакт состоял из двух частей — открытой и секретной. \textbf{Открытым для общественности был пакт о ненападении}, стороны которого обязывались воздерживаться от нападения друг на друга. Кроме того, в случае нападения третьего государства на одну из сторон вторая обязывалась не оказывать никакого содействия этому третьему государству. Участники договорились не заключать никаких соглашений с другими державами, направленных против второй стороны. Пакт о ненападении был заключен сроком на десять лет и подлежал автоматическому продлению еще на пять лет, если ни одна из сторон не выказывала заинтересованности в прекращении его действия.

\hfill

\textbf{Секретная часть пакта} представляла собой протокол, определявший советскую и немецкую сферы влияния в Восточной Европе. Эстония, Латвия и Бессарабия попали в советскую сферу. Граница интересов участников в Польше прошла по рекам Нарев, Висла и Сан.

\subsubsection{Германо-советский пакт в действии}

После вступления пакта Молотова-Риббентропа в силу Германия, не опасаясь советского вмешательства, \textbf{1 сентября 1939 года напала на Польшу}. 3 сентября 1939 года Великобритания и Франция, пятью месяцами ранее предоставившие Польше гарантии защиты ее границ, объявили войну Германии. Две недели спустя, 17 сентября, советская армия вторглась в Польшу с востока. Эти события положили начало Второй мировой войне.

\hfill

\textbf{Германия и Советский Союз установили контроль над сферами влияния, зафиксированными в секретном протоколе к пакту о ненападении}. В протокол были внесены изменения: Литва и город Вильнюс (на тот момент — польский город Вильно) включены в советскую сферу, закреплена новая германо-советская граница на территории оккупированной Польши. 29 \textbf{сентября 1939 года был закреплен раздел Польши}. Германия оккупировала западную и значительную часть центральной Польши. Западные провинции были включены непосредственно в состав рейха. Остальную часть Польши оккупировал и аннексировал Советский Союз.

\hfill

В соответствии с договором Советский Союз также захватил  другие территории в своей сфере влияния. \textbf{30 ноября 1939 г. СССР напал на Финляндию}. Война продлилась четыре месяца, была присоединена территория Финляндии вдоль советской границы, в частности, область рядом с Ленинградом (современный Санкт-Петербург). Летом 1940 года СССР оккупировал и аннексировал страны Балтии — Эстонию, Латвию и Литву. Также были захвачены румынские области Северная Буковина и Бессарабия.

\subsubsection{Конец германо-советского пакта}

\textbf{Гитлер считал германо-советский пакт о ненападении всего лишь временным средством, тактическим маневром}. Он с самого начала не собирался соблюдать пакт в течение десяти лет. В долгосрочные планы Гитлера всегда входило нападение германской армии на Советский Союз и создание на захваченных территориях «жизненного пространства» (Lebensraum) для немцев. \textbf{Однако перед этим Гитлер намеревался подчинить Польшу и нанести поражение Франции и Великобритании. Пакт о ненападении позволил ему провести эти операции, не опасаясь агрессии СССР и войны на два фронта}. 

\pagebreak
\subsection{Советско-финская война: замысел и результаты}

Статья по теме: \url{https://istoriarusi.ru/cccp/sovetsko-finskaya-vojna-zimnaya-1939-1940.html}

\subsubsection{Замысел}

Советско-финская война 1939-1940 года (Советско-финская война, в Финляндии известна как Зимняя война) — вооружённый конфликт между СССР и Финляндией в период с 30 ноября 1939 года по 12 марта 1940 года.

\hfill

Его причиной стало желание советского руководства отодвинуть финскую границу от Ленинграда (ныне Санкт-Петербург) с целью укрепления безопасности северо-западных границ СССР, и отказ финской стороны сделать это. Советское правительство просило предоставить в аренду части полуострова Ханко и некоторых островов в Финском заливе в обмен на большую по площади советскую территорию в Карелии с последующим заключением договора о взаимопомощи.

\hfill

Финское правительство считало, что принятие советских требований ослабит стратегические позиции государства, приведет к утрате Финляндией нейтралитета и ее подчинению СССР. Советское руководство, в свою очередь, не желало отказываться от своих требований, необходимых, по его мнению, для обеспечения безопасности Ленинграда.

\subsubsection{Результаты}

СССР 28 февраля передало Финляндии свои условия для заключения мира. Сами переговоры проходили в Москве 8 - 12 марта. После этих переговоров советско-финская война закончилась 12 марта 1940 года. Условия мира были следующими:

\begin{enumerate}
    \item СССР получал Карельский перешеек вместе с Выборгом (Виипури), заливом и островами.
    \item Западное и Северное побережье Ладожского озера, вместе с городами Кексгольм, Суоярви и Сортавала.
    \item Острова в Финском заливе.
    \item Остров Ханко с морской территорией и базой сдавался в аренду СССР на 50 лет. За аренду ежегодно СССР платило 8 миллионов немецких марок.
    \item Договор между Финляндией и СССР от 1920 года утратил свою силу.
    \item С 13 марта 1940 года прекращаются боевые действия.
\end{enumerate}

Советско-финская война 1939-1940 годов даже при кратком изучении указывает как на абсолютно негативные, так и на абсолютно позитивные моменты. \textbf{Негатив – кошмар первых месяцев войны и огромное количество жертв}. По большому счету именно декабрь 1939 и начало января 1940 продемонстрировали всему миру, что советская армия слаба. Так оно и было на самом деле. Но в этом \textbf{был и позитивный момент: советское руководство увидело реальную силу своей армии}. Нам с детства говорят, что красная армия чуть ли не с 1917 года была сильнейшей в мире, но это крайне далеко от реальности. Единственное крупное испытание этой армии – Гражданская война.

\hfill

Перед войной с Финляндией \textbf{руководство СССР витало в облаках, полагая, что имеет сильную армию}. Но декабрь 1939 года показал, что это не так. Армия была крайне слаба. \textbf{Но начиная с января 1940 года вносились изменения (кадровые и организационные), которые изменили ход войны, и которые во многом подготовили боеспособную армию для Отечественной войны}. Доказать это очень легко. Практически весь декабрь 39-го РККА штурмовала линию Маннергейма – результата не было. 11 февраля 40-го года линия Маннергейма была прорвана за 1 день. Этот прорыв был возможен, поскольку его осуществляла уже другая армия, более дисциплинированная, организованная, обученная.

\subsubsection{Потери}

По современным данным, \textbf{потери финской армии оцениваются в примерно 26 тысяч человек погибших и пропавших без вести}

\textbf{Потери советской армии оцениваются в 126 875 убитых и умерших на этапах санитарной эвакуации}

\pagebreak
\subsection{«Вхождение» Прибалтики и Бессарабии в состав СССР}

\subsubsection{Присоединение Бессарабии и Северной Буковины к СССР}

К началу Второй мировой войны территория Бессарабии являлась частью Королевства Румыния, однако правительство Советского Союза не признавало этого и планировало разрешить данный вопрос военным путём[4]. Так 23 августа 1939 года между Нацистской Германией и Советским Союзом был заключён договор о ненападении с секретным протоколом к нему, по которому Третий Рейх уступал Бессарабию в пользу «сферы влияния» СССР[5][6][7].

\hfill

\textbf{Реализация секретного протокола началась 26 июня 1940}, когда министр иностранных дел СССР Вячеслав Молотов вручил румынскому послу в Москве Георге Давидеску заявление советского правительства, в котором выдвигался ультиматум с требованием передать СССР Бессарабии и северной части Буковины в границах согласно приложенной карте.

\hfill

В ответ на ультиматум Москвы, \textbf{27 июня 1940 года, в Румынии была объявлена всеобщая мобилизация}, но незадолго до планируемого начала военной операции, после вручения ноты советской стороной посланнику Королевства Румыния в СССР Г. Давидеску и консультаций с официальными представителями от Германии, Италии и стран Балканской Антанты, \textbf{королём Румынии Каролем II было принято решение об удовлетворении требования о передаче Бессарабии и Северной Буковины Советскому Союзу}[8][9].

\hfill

Операция по занятию спорных территорий Красной армией началась \textbf{28 июня 1940 года} и продлилась 6 дней. Немногим позднее — \textbf{2 августа 1940 была образована Молдавская ССР} со столицей в Кишинёве (до этого город находился в Бессарабии), были основаны Черновицкая и Аккерманская области УССР, а в состав Одесской области было включено пять районов ранее входивших в состав Молдавской АССР[10][11].

\subsubsection{Присоединение Прибалтики к СССР}

\textbf{Присоединение Прибалтики к СССР} (в странах Балтии и многих других эти события и последующий период нахождения этих стран в составе СССР называются советской оккупацией) — включение независимых прибалтийских государств — Эстонии, Латвии и Литвы — в состав СССР, ставшее следствием подписания СССР и нацистской Германией в августе 1939 года договора о ненападении между Германией и Советским Союзом и Договора о дружбе и границе, секретные протоколы которых зафиксировали разграничение сфер интересов этих двух держав в Восточной Европе.

\paragraph{Ввод советских войск 1939 года}

Статьи: \href{https://ru.wikipedia.org/wiki/%D0%9F%D0%B0%D0%BA%D1%82_%D0%BE_%D0%B2%D0%B7%D0%B0%D0%B8%D0%BC%D0%BE%D0%BF%D0%BE%D0%BC%D0%BE%D1%89%D0%B8_%D0%BC%D0%B5%D0%B6%D0%B4%D1%83_%D0%A1%D0%A1%D0%A1%D0%A0_%D0%B8_%D0%9B%D0%B0%D1%82%D0%B2%D0%B8%D0%B9%D1%81%D0%BA%D0%BE%D0%B9_%D0%A0%D0%B5%D1%81%D0%BF%D1%83%D0%B1%D0%BB%D0%B8%D0%BA%D0%BE%D0%B9}{Пакт о взаимопомощи между СССР и Латвийской Республикой}, \href{https://ru.wikipedia.org/wiki/%D0%94%D0%BE%D0%B3%D0%BE%D0%B2%D0%BE%D1%80_%D0%BE_%D0%BF%D0%B5%D1%80%D0%B5%D0%B4%D0%B0%D1%87%D0%B5_%D0%9B%D0%B8%D1%82%D0%BE%D0%B2%D1%81%D0%BA%D0%BE%D0%B9_%D1%80%D0%B5%D1%81%D0%BF%D1%83%D0%B1%D0%BB%D0%B8%D0%BA%D0%B5_%D0%B3%D0%BE%D1%80%D0%BE%D0%B4%D0%B0_%D0%92%D0%B8%D0%BB%D1%8C%D0%BD%D0%BE_%D0%B8_%D0%92%D0%B8%D0%BB%D0%B5%D0%BD%D1%81%D0%BA%D0%BE%D0%B9_%D0%BE%D0%B1%D0%BB%D0%B0%D1%81%D1%82%D0%B8_%D0%B8_%D0%BE_%D0%B2%D0%B7%D0%B0%D0%B8%D0%BC%D0%BE%D0%BF%D0%BE%D0%BC%D0%BE%D1%89%D0%B8_%D0%BC%D0%B5%D0%B6%D0%B4%D1%83_%D0%A1%D0%BE%D0%B2%D0%B5%D1%82%D1%81%D0%BA%D0%B8%D0%BC_%D0%A1%D0%BE%D1%8E%D0%B7%D0%BE%D0%BC_%D0%B8_%D0%9B%D0%B8%D1%82%D0%B2%D0%BE%D0%B9}{Договор о передаче Литовской республике города Вильно и Виленской области и о взаимопомощи между Советским Союзом и Литвой}, \href{https://ru.wikipedia.org/wiki/%D0%9F%D0%B0%D0%BA%D1%82_%D0%BE_%D0%B2%D0%B7%D0%B0%D0%B8%D0%BC%D0%BE%D0%BF%D0%BE%D0%BC%D0%BE%D1%89%D0%B8_%D0%BC%D0%B5%D0%B6%D0%B4%D1%83_%D0%A1%D0%A1%D0%A1%D0%A0_%D0%B8_%D0%AD%D1%81%D1%82%D0%BE%D0%BD%D0%B8%D0%B5%D0%B9}{Пакт о взаимопомощи между СССР и Эстонией}

Ограниченный контингент Красной Армии (например, в Латвии в приложенном к договору о взаимопомощи конфиденциальном протоколе была согласована численность советских войск в 25 тыс. человек[9], что сопоставимо с численностью армии Латвии) был введён с разрешения президентов балтийских стран, и были заключены соглашения \textbf{под предлогом защиты балтийских стран от нацистской угрозы}.

\paragraph{Ультиматумы лета 1940 года и смещение прибалтийских правительств}

\textbf{3 июня поверенный в делах СССР в Литве В. Семёнов пишет обзорную записку о положении в Литве}, в которой советское полпредство обращало внимание Москвы на стремление правительства Литвы «предаться в руки Германии», и на активизацию «деятельности германской пятой колонны и вооружение членов союза стрелков», подготовку к мобилизации

\hfill

\textbf{4 июня} под видом учений войска Ленинградского, Калининского и Белорусского Особого военных округов были подняты по тревоге и начали выдвижение к границам прибалтийских государств. 

\textbf{14 июня} советское правительство предъявило ультиматум Литве, а \textbf{16 июня} — Латвии и Эстонии. В основных чертах смысл ультиматумов совпадал — \textbf{правительства этих государств обвинялись в грубом нарушении условий ранее заключенных с СССР Договоров о взаимопомощи}, и выдвигалось требование сформировать правительства, способные обеспечить выполнение этих договоров, а также допустить на территорию этих стран дополнительные контингенты войск. \textbf{Условия были приняты}

\hfill

15 июня дополнительные контингенты советских войск были введены в Литву, а 17 июня — в Эстонию и Латвию.

\hfill

Литовский президент Антанас Сметона настаивал на организации сопротивления советским войскам, однако, получив отказ большей части правительства, бежал в Германию, а его латвийский и эстонский коллеги — Карлис Улманис и Константин Пятс — пошли на сотрудничество с новой властью, как и литовский премьер Антанас Меркис. \textbf{Во всех трёх странах были сформированы дружественные СССР правительства}.

\paragraph{Вхождение прибалтийских государств в СССР}

Новые правительства сняли запреты на деятельность коммунистических партий и проведение демонстраций, выпустили просоветски настроенных политзаключенных и назначили внеочередные парламентские выборы. \textbf{На голосованиях, состоявшихся 14 июля во всех трёх государствах, формально одержавшие победу прокоммунистические Блоки (Союзы) трудового народа были единственными избирательными списками, допущенными к выборам}. 

\hfill

Вновь избранные \textbf{парламенты уже 21—22 июля провозгласили создание Эстонской ССР, Латвийской ССР и Литовской ССР и приняли Декларации о вхождении в СССР}. 3—6 августа 1940 года, в соответствии с решениями Верховного Совета СССР, эти республики были приняты в состав Советского Союза. 

\pagebreak
\subsection{Советско-германские переговоры в ноябре 1940 года и возможность присоединения СССР к «Антикоминтерновскому пакту»}

\textbf{Пакт четырёх держав Оси} — проект договора о дружбе и экономической поддержке с возможностью ведения совместных боевых действий против других стран, который планировалось заключить между СССР и странами Оси в конце 1940 года. \textbf{Пакт имел целью создание мощного военно-политического альянса и фактического военного распределения Восточного полушария планеты между странами-подписантами}. Название договора отражает количество договаривающихся сторон, которыми должны были стать нацистская Германия, фашистская Италия, милитаристская Япония и СССР. \textbf{Подписание Пакта не состоялось из-за чрезмерных встречных требований СССР, оказавшихся неприемлемыми для Германии}

\subsubsection{Предпосылки}

\textbf{В сентябре 1940 года странами Оси был подписан Тройственный пакт}, согласно которому нацистская Германия, фашистская Италия и Японская империя обязались помогать друг другу в установлении «нового мирового порядка». \textbf{В конце сентября 1940 года Гитлер направил послание Сталину, известив его о подписании Тройственного пакта, а позднее предложил Советскому Союзу окончательно присоединиться к странам Оси и принять участие в дележе «английского наследства» в Иране и Индии}. При этом в Тройственный пакт (в отличие от заключённого ранее Антикоминтерновского пакта) не была включена франкистская Испания, к которой у СССР было негативное отношение после участия советских «добровольцев» в гражданской войне в Испании.

\hfill

\textbf{12 ноября 1940 года Адольф Гитлер предложил Вячеславу Молотову, который находился с визитом в Берлине, присоединение СССР к странам Оси в качестве полноправного четвёртого участника}. Проект Договора разрабатывался Иоахимом фон Риббентропом и был зачитан Молотову 13 ноября 1940 года в бомбоубежище, во время бомбардировки Берлина английской авиацией. 

\subsubsection{Пожелания СССР по договору}

\textbf{СССР согласился принять условия проекта пакта четырёх держав об их политическом сотрудничестве и экономической взаимопомощи}, изложенных Риббентропом в беседе с Молотовым, \textbf{но выставил свои условия, которые оказались неприемлемы для Германии}:

\begin{enumerate}
    \item Если немецкие войска будут немедленно выведены из Финляндии, представляющей сферу влияния СССР согласно Пакту Риббентропа-Молотова
    \item Если будет обеспечена безопасность СССР в проливах путём заключения пакта взаимопомощи между СССР и Болгарии и организации военной и военно-морской базы СССР в районе Босфора и Дарданелл «на началах долгосрочной аренды»;
    \item Если вектором сферы интересов СССР будет признан район к югу от Батуми и Баку в общем направлении к Персидскому заливу;
    \item Если Япония откажется от своих концессионных прав по углю и нефти на Северном Сахалине «на условиях справедливой компенсации».
\end{enumerate}

\subsubsection{Последствия}

О \textbf{серьёзности и упущенной возможности} взаимных намерений Германии и СССР (относительно заключения данного Пакта) свидетельствуют факты того, что как Германия настойчиво предлагала такой союз в 1939—1940 гг, так и СССР ждал ответа на свои встречные предложения, которые однако, оказались чрезмерны и неприемлемы для Германии.

\textbf{При первой же беседе с Гитлером новый посол СССР в Германии В. Г Деканозов высказал пожелание продолжить обсуждение этой темы}. Риббентроп ответил, что советские предложения обсуждаются с Италией и Японией, однако ответ так и не был дан, поскольку увеличение военного присутствия СССР в Южной Европе было для стран Оси неприемлемым с точки зрения безопасности, на что немецкая сторона неоднократно намекала во время переговоров.

О заинтересованности СССР в заключении Пакта говорит и то обстоятельство, что \textbf{одновременно с изложением пожеланий к проекту Договора, советской стороной были сделаны дипломатические шаги в направлении Болгарии}. А именно была организована «случайная встреча» Молотова с болгарским послом Стаменовым. Одновременно со специальной миссией посетил Софию и генеральный секретарь наркомата иностранных дел СССР Соболев. О смысле этой миссии Сталин сообщал Коминтерну, целью которого была мировая большевистская революция: «Мы сегодня делаем болгарам предложение о заключении пакта взаимопомощи».

\hfill

Третий Рейх молчаливо отверг предложения СССР и через месяц Гитлер принял к реализации план «Барбаросса», а советская военная разведка начала сообщать в Москву о немецких действиях, свидетельствующих о приготовлении к нападению Германии на Советский Союз. 

\pagebreak
\subsection{Причины обострения советско-германских отношений в конце 1940 – начале 1941 г.}

\end{document}
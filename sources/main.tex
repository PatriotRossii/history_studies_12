\documentclass{article}
\usepackage[utf8]{inputenc}

\usepackage[T2A]{fontenc}
\usepackage[utf8]{inputenc}
\usepackage[russian]{babel}

\usepackage{multienum}
\usepackage{geometry}
\usepackage{hyperref}

\geometry{
    left=1cm,right=1cm,
    top=2cm,bottom=2cm
}

\usepackage{graphicx}
\graphicspath{ {./images/} }

\title{История}
\author{Лисид Лаконский}
\date{May 2023}

\newtheorem{definition}{Определение}

\begin{document}
\raggedright

\maketitle
\tableofcontents
\pagebreak

\section{Практическое занятие по истории №13, «СССР в канун Второй мировой войны»}

\subsection{Нарастание напряженности в Европе после прихода в 1933 г. в Германии к власти НСДАП}

\pagebreak
\subsection{Проба сил перед столкновением. Гражданская война в Испании и великие державы}

\pagebreak
\subsection{Сближение СССР и Германии. Пакт Риббентропа — Молотова и раздел Европы}

\pagebreak
\subsection{Советско-финская война: замысел и результаты}

\pagebreak
\subsection{«Вхождение» Прибалтики и Бессарабии в состав СССР}

\pagebreak
\subsection{Советско-германские переговоры в ноябре 1940 года и возможность присоединения СССР к «Антикоминтерновскому пакту»}

\pagebreak
\subsection{Причины обострения советско-германских отношений в конце 1940 – начале 1941 г.}

\end{document}